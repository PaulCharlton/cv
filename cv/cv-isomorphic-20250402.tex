%-------------------------
% Resume in Latex
% Author : Jake Gutierrez
% Based off of: https://github.com/sb2nov/resume
% License : MIT
%------------------------

\documentclass[letterpaper,11pt]{article}

\usepackage{latexsym}
\usepackage{changepage}  % Provides adjustwidth environment
\usepackage[empty]{fullpage}
\usepackage{titlesec}
\usepackage{marvosym}
\usepackage[usenames,dvipsnames]{color}
\usepackage{verbatim}
\usepackage{enumitem}
\usepackage[hidelinks]{hyperref}
\usepackage{fancyhdr}
\usepackage[english]{babel}
\usepackage{tabularx}

\usepackage{fontawesome}
\usepackage{multicol}
\setlength{\multicolsep}{-3.0pt}
\setlength{\columnsep}{-1pt}
\usepackage{fontspec}
\usepackage{emoji}
%----------FONT OPTIONS----------
% sans-serif
% \usepackage[sfdefault]{FiraSans}
% \usepackage[sfdefault]{roboto}
% \usepackage[sfdefault]{noto-sans}
% \usepackage[default]{sourcesanspro}

% serif
% \usepackage{CormorantGaramond}
% \usepackage{charter}


\pagestyle{fancy}
\fancyhf{} % clear all header and footer fields
\fancyfoot{}
\renewcommand{\headrulewidth}{0pt}
\renewcommand{\footrulewidth}{0pt}

% Adjust margins
\addtolength{\oddsidemargin}{-0.6in}
\addtolength{\evensidemargin}{-0.5in}
\addtolength{\textwidth}{1.19in}
\addtolength{\topmargin}{-.7in}
\addtolength{\textheight}{1.4in}

\urlstyle{same}

\raggedbottom
\raggedright
\setlength{\tabcolsep}{0in}

% Sections formatting
\titleformat{\section}{
    \vspace{-4pt}\scshape\raggedright\large\bfseries
}{}{0em}{}[\color{black}\titlerule \vspace{-5pt}]

%-------------------------
% Custom commands
\newcommand{\resumeItem}[1]{
    \item\small{
            {#1 \vspace{-2pt}}
    }
}

\newcommand{\classesList}[4]{
    \item\small{
            {#1 #2 #3 #4 \vspace{-2pt}}
    }
}

\newcommand{\resumeSubheading}[4]{
    \vspace{-2pt}\item
    \begin{tabular*}{1.0\textwidth}[t]{l@{\extracolsep{\fill}}r}
    \textbf{#1}  \textit{\small#3} & \textbf{\small #2} \\
    \end{tabular*}\vspace{-7pt}
}

\newcommand{\resumeSubSubheading}[2]{
    \item
    \begin{tabular*}{0.97\textwidth}{l@{\extracolsep{\fill}}r}
    \textit{\small#1} & \textit{\small #2} \\
    \end{tabular*}\vspace{-7pt}
}

\newcommand{\resumeProjectHeading}[2]{
    \item
    \begin{tabular*}{1.001\textwidth}{l@{\extracolsep{\fill}}r}
    \small#1 & \textbf{\small #2}\\
    \end{tabular*}\vspace{-7pt}
}

\newcommand{\resumeSubItem}[1]{\resumeItem{#1}\vspace{-4pt}}

\renewcommand\labelitemi{$\vcenter{\hbox{\tiny$\bullet$}}$}
\renewcommand\labelitemii{$\vcenter{\hbox{\tiny$\bullet$}}$}

\newcommand{\resumeSubHeadingListStart}{\begin{itemize}[leftmargin=0.0in, label={}]}
\newcommand{\resumeSubHeadingListEnd}{\end{itemize}}
\newcommand{\resumeItemListStart}{\begin{itemize}}
\newcommand{\resumeItemListEnd}{\end{itemize}\vspace{-5pt}}

% Patch section to include right-aligned subtitle
\newcommand{\sectionwithnote}[2]{%
\section[#1]{%
\noindent
\makebox[\textwidth]{\textbf{\Large #1} \hfill \textit{#2}}%
}%
}

%-------------------------------------------
%%%%%%  RESUME STARTS HERE  %%%%%%%%%%%%%%%%%%%%%%%%%%%%


\begin{document}

%----------HEADING----------
% \begin{tabular*}{\textwidth}{l@{\extracolsep{\fill}}r}
%   \textbf{\href{http://sourabhbajaj.com/}{\Large Sourabh Bajaj}} & Email : \href{mailto:sourabh@sourabhbajaj.com}{sourabh@sourabhbajaj.com}\\
%   \href{http://sourabhbajaj.com/}{http://www.sourabhbajaj.com} & Mobile : +1-123-456-7890 \\
% \end{tabular*}

\begin{center}
{\Huge \scshape Paul Charlton} \\ \vspace{4pt}

\href{mailto:techguru@byiq.com}{\raisebox{-0.2\height}\faEnvelope\ } \quad
\href{https://github.com/PaulCharlton}{\raisebox{-0.2\height}\faGithub\ } \quad
\href{https://gitlab.com/techguru}{\raisebox{-0.2\height}\faGitlab\ } \quad
\href{https://www.linkedin.com/in/techguru/}{\raisebox{-0.2\height}\faLinkedin\ } \quad
\href{https://paul-charlton.vercel.app/}{\raisebox{-0.2\height}\faGlobe\ } \quad
\href{https://en.wikipedia.org/wiki/Paul_Charlton_(technologist)}{\raisebox{-0.2\height}\faWikipediaW\ } \quad
\href{tel:+18082787904}{\raisebox{-0.2\height}\faPhone\ } \quad
\href{https://signal.me/#p/+18082787904}{\raisebox{-0.2\height}\faComment\ } \quad
\href{https://wa.me/18082787904}{\raisebox{-0.2\height}\faWhatsapp\ }

\vspace{-8pt}
\end{center}



%-----------EDUCATION-----------
\sectionwithnote{EDUCATION}{}
\resumeSubHeadingListStart
\resumeSubheading
{Rensselaer Polytechnic Institute}
{Sept. 1984 -- May 1988}
{BS: Electrical, Computer, and Systems Engineering}
{Troy, New York}
\resumeSubHeadingListEnd

\sectionwithnote{SUMMARY}{Trusted C-Level advisor; Distinguished Engineer with C-Level Impact}
\begin{itemize}
\item{
Distinguished Engineer and platform strategist operating as a high-leverage individual contributor with consistent C-level business impact. Entrusted by Fortune 500 executives to own and deliver mission-critical infrastructure, platform modernization, and developer experience transformation at scale.
}
\item{
Architect behind foundational technologies that power over \$1T in ARR — including Apple QuickTime, Intuit QuickBooks, and core JVM graphics. Presented on the WWDC main stage and held pivotal roles at Apple, Sun, and Intuit. Now focused on driving end-to-end architecture — from IoT drivers to cloud-native SaaS — with a passion for clean systems and ruthless simplicity.
}
\item{
Recognized force multiplier: delivered 100x gains in reliability, latency, and developer productivity through deep systems expertise, automation, and hands-on mentorship. Led org-wide transformations in companies of 2,000+, directly managing teams up to 200.
}
\item{
Award-winning inventor with multiple patents and a sharp business instinct for IP protection — reclaiming over \$150M in license revenue. Trusted partner in strategy, innovation, and delivery. Former Chief Cloud Officer, CTO, and VP Engineering; now advising VC-backed startups while remaining fully IC by choice.
}
\end{itemize}


%-----------EXPERIENCE-----------
\sectionwithnote{EXPERIENCE}{curated for relevance to Isomorphic Labs}
\resumeSubHeadingListStart

\resumeSubheading
{\href{https://www.virginia.edu/}{University of Virginia - School of Medicine~\faExternalLink}
\quad\small
\textit
{\href {https://paul-charlton.vercel.app/story/9}{[Full Story]~{\fontspec{Symbola}\symbol{"1F517}}}}
}
{1979 - 1986}
{Researcher}
{Charlottesville, VA}
\resumeItem{
Youngest paid researcher at UVA School of Medicine at 14, recruited after scoring “>170 IQ off the chart” in UVA’s gifted research program.  In a vision research lab under the Department of Physiology, I pioneered the lab-scale integration of analog biomedical equipment with digital computers, automating data collection and experimental workflows in a fully dark lab. I designed and built custom circuits, developed 6502 assembly drivers for GPIB-controlled systems, and implemented an early speaker-independent voice recognition system - advancing published medical research and mentoring graduate-level scientists before entering college.
}

\resumeSubheading
{\href{https://www.deshaw.com}{D.E. Shaw \& Co.~\faExternalLink}
\quad\small
\textit
{\href {https://paul-charlton.vercel.app/story/25}{[Full Story]~{\fontspec{Symbola}\symbol{"1F517}}}}
}
{November 1992 -- April 1994}
{Technology Specialist}
{New York, NY}
\resumeItem{
At D. E. Shaw, I joined as employee \#24 and led engineering for Jeff Bezos’s Third Market trading desk, architecting a GUI-based order entry system and scaling infrastructure to support market making in 2,000+ securities. I diagnosed critical system deadlocks in the Stratus OS, optimized predictive trading algorithms using multidimensional statistical calculus, and pioneered in-house logging of historical market data to replace costly external sources. I contributed to early brainstorming sessions that seeded the idea for what became Amazon. My push for employee equity — tying ownership to impact rather than seniority — challenged the status quo and ultimately helped catalyze an exodus that included Jeff’s departure to launch it.
}

\resumeSubheading
{\href{https://www.apple.com}{Apple Computer Inc.~\faExternalLink}
\quad\small
\textit
{\href{https://paul-charlton.vercel.app/story/11}{[Full Story]~{\fontspec{Symbola}\symbol{"1F517}}}}
}
{April 1994 -- June 1996}
{Principal Engineer}
{Cupertino, CA}
\resumeItem{
I led the cross-platform port of QuickTime and macOS from proprietary Apple hardware to Windows (x86) by building a fully preemptive, thread-safe implementation of the Macintosh Toolbox from scratch and integrating QuickDraw 2D. I architected and launched the QuickTime Media Layer (QTML), which enabled Mac applications to run on Windows and SGI IRIX. QTML later formed the foundation of Apple’s Carbon API and macOS X transition. My work was pivotal to Apple’s \$150M legal victory over Microsoft/Intel, and I presented the cross-platform QuickTime rollout at WWDC 1996 in the main hall to over 1,000 developers.
}

\resumeSubheading
{\href{https://www.chase.com}{Chase Manhattan Bank~\faExternalLink}
\quad\small
\textit{
\href{https://paul-charlton.vercel.app/story/17}{[Full Story]~{\fontspec{Symbola}\symbol{"1F517}}}}
}
{June 1990 -- November 1992}
{Director of Advanced Technology Lab}
{New York, NY}
\resumeItem{
I modernized Chase’s retail and back-office banking systems, replacing paper-based workflows with PC and networked solutions that drastically reduced risk, cut processing time from days to hours, and supported scalable acquisitions. I architected and developed the first ever Windows GUI-based online banking platform, engineered high-security systems for wire transfers and vault reconciliation, and implemented middleware to unify legacy systems through screen-scraping and field validation. My efforts enhanced operational security, uncovered fraud, and saved millions through alternative computing strategies and vendor negotiations.
}

\resumeSubheading
{\href{https://www.oracle.com/sun}{Sun Microsystems~\faExternalLink}
\quad\small
\textit{
\href{https://paul-charlton.vercel.app/story/4}{[Full Story]~{\fontspec{Symbola}\symbol{"1F517}}}}
}
{2008}
{Principal Consulting Engineer}
{Santa Clara, CA}
\resumeItem{
In under 10 weeks, I led a 10-person team to rescue and deliver JavaFX 1.0 on time—rewriting the multimedia pipeline, eliminating 30\% of the legacy code, and adding 40,000 lines of high-performance, test-driven code to enable smooth, synchronized video playback across platforms. I re-architected the system for native performance, preemptive multithreading, and customer delight, coordinating closely with product management and external partners. Our release on Dec 4, 2008 marked Sun’s first on-time software launch in over a decade and heralded a 51\% increase in market cap within 15 days.
}
\newpage
\resumeSubheading
{\href{https://www.oracle.com/sun}{Sun Microsystems - JavaSoft~\faExternalLink}
\quad\small
\textit{
\href{https://paul-charlton.vercel.app/story/5}{[Full Story]~{\fontspec{Symbola}\symbol{"1F517}}}}
}
{1997}
{Principal Consulting Engineer}
{Palo Alto, CA}
\resumeItem{
As Principal Consulting Engineer, I rearchitected and optimized the Java2D subsystem for JDK 2, boosting glyph rendering performance by 40x to deliver over 200,000 glyphs/sec on 300Mhz Pentium II. I also applied low-level Intel i386 pipelining techniques to the Java Bytecode interpreter, achieving a 5x performance gain. This foundational code remains virtually unchanged in every Android device today and across graphical Java-based systems globally.
}

\resumeSubheading
{\href{https://www.hp.com}{Hewlett-Packard~\faExternalLink}
\quad\small
\textit{
\href{https://paul-charlton.vercel.app/story/38}{[Full Story]~{\fontspec{Symbola}\symbol{"1F517}}}}
}
{June 1988 -- May 1990}
{Hardware and Software Engineer}
{Greeley, CO}
\resumeItem{
At Hewlett-Packard, I developed advanced SCSI-II drivers and boot ROMs for magneto-optic and tape storage devices across HP-UX, MPE, and Apollo Domain OS, including debugging low-level OS synchronization flaws and pioneering in-house ASIC/VLSI chip design with a successful first tape-out. I led a data recovery initiative for NASA's historic space mission telemetry, creating custom software and decoding pipelines for degraded tape reels, and contributed to integrating HP storage with Apollo workstations post-acquisition. Certified as a Six Sigma coach, I worked cross-functionally to enforce defect prevention at the requirements stage and applied HP's culture of merit-driven, engineer-led innovation.
}

\resumeSubHeadingListEnd
\vspace{-16pt}

\section*{Publications, Patents and Products}

\textbf{Key Products:} Apple QuickTime (iTunes, iOS), Apple Carbon API, JavaFX, Java2D (Android), MDOS, QuickBooks

\vspace{0.5em}
\textbf{Selected Publications:}
\noindent
Morton, R. W., Chung, J. K., Miller, J. L., Charlton, J. P., \& Fager, R. S. (1986). \textit{Extended sensitivity for the calcium selective electrode}. Analytical Biochemistry, 157(2), 345–352. \href{https://scholar.google.com/scholar_lookup?title=Extended+sensitivity+for+the+calcium+selective+electrode&author=Morton&publication_year=1986&journal=Analytical+Biochemistry&volume=157&pages=345-352&doi=10.1016/0003-2697(86)90636-6}\newline
{Google Scholar~\faExternalLink}.  \href{https://doi.org/10.1016/0003-2697(86)90636-6}{DOI: 10.1016/0003-2697(86)90636-6}

\vspace{0.5em}
\textbf{Selected Patents:}

\textbf{\href{https://patents.google.com/patent/US5875354}{US 5,875,354~\faExternalLink}} (1999) – \textit{System for Synchronization by Modifying the Rate of Conversion by Difference of Rate Between First Clock and Audio Clock}
\textbullet{} Inventors: Paul Charlton, Keith Gurganus; Assignee: Apple Inc.; Early solution to audio/video sync via dynamic sample rate conversion. Now core to AV streaming, DAWs, AI video rendering, and IoT synchronization.

\vspace{0.5em}

\textbf{\href{https://patents.google.com/patent/US5825359}{US 5,825,359~\faExternalLink}} (1998) – \textit{Improved Arbitration of a Display Screen in a Computer System}
\textbullet{} Queue-based screen access control in multi-process environments.; Anticipated today’s GPU compositing (Quartz, DWM, Wayland) and virtual desktop rendering.

\vspace{0.5em}

\textbf{\href{https://patents.google.com/patent/US5949434}{US 5,949,434~\faExternalLink}} (1999) – \textit{Scaling 2D Graphic Images Without Banding Artifacts}
\textbullet{} Enhances Bresenham’s algorithm with pseudo-random shifts to reduce visual artifacts.; Forefather of adaptive upscaling in DLSS, medical imaging, and AI-based super-resolution.

%-----------PROJECTS-----------
% \section{Projects}
%     \vspace{-5pt}
%     \resumeSubHeadingListStart
%       \resumeProjectHeading
%           {\textbf{Gym Reservation Bot} $|$ \emph{Python, Selenium, Google Cloud Console}}{January 2021}
%           \resumeItemListStart
%             \resumeItem{Developed an automatic bot using Python and Google Cloud Console to register myself for a timeslot at my school gym.}
%             \resumeItem{Implemented Selenium to create an instance of Chrome in order to interact with the correct elements of the web page.}
%             \resumeItem{Created a Linux virtual machine to run on Google Cloud so that the program is able to run everyday from the cloud.}
%             \resumeItem{Used Cron to schedule the program to execute automatically at 11 AM every morning so a reservation is made for me.}
%           \resumeItemListEnd
%           \vspace{-13pt}
%       \resumeProjectHeading
%           {\textbf{Ticket Price Calculator App} $|$ \emph{Java, Android Studio}}{November 2020}
%           \resumeItemListStart
%             \resumeItem{Created an Android application using Java and Android Studio to calculate ticket prices for trips to museums in NYC.}
%             \resumeItem{Processed user inputted information in the back-end of the app to return a subtotal price based on the tickets selected.}
%             \resumeItem{Utilized the layout editor to create a UI for the application in order to allow different scenes to interact with each other.}
%           \resumeItemListEnd
%           \vspace{-13pt}
%           \resumeProjectHeading
%           {\textbf{Transaction Management GUI} $|$ \emph{Java, Eclipse, JavaFX}}{October 2020}
%           \resumeItemListStart
%             \resumeItem{Designed a sample banking transaction system using Java to simulate the common functions of using a bank account.}
%             \resumeItem{Used JavaFX to create a GUI that supports actions such as creating an account, deposit, withdraw, list all acounts, etc.}
%             \resumeItem{Implemented object-oriented programming practices such as inheritance to create different account types and databases.}
%           \resumeItemListEnd
%     \resumeSubHeadingListEnd
% \vspace{-15pt}


%
%-----------PROGRAMMING SKILLS-----------
\sectionwithnote{Technologies \& Keywords}{Machine Readable}
\fontsize{3pt}{4pt}\selectfont{
\begin{itemize}[leftmargin=0.1in, label={}, itemsep=0pt, parsep=0pt, topsep=0pt, partopsep=0pt]
\item
\textbf{\underline{Hardware Interfaces:}} \hspace{0pt}
AGP - Accelerated Graphics Port;
Centronics - Centronics Parallel Interface;
EISA - Extended Industry Standard Architecture;
eSATA - External Serial ATA;
FireWire - IEEE 1394 High Performance Serial Bus;
GPIB - General Purpose Interface Bus;
HDMI - High-Definition Multimedia Interface;
HPIB - Hewlett-Packard Interface Bus;
ISA - Industry Standard Architecture;
NuBus - New Bus Architecture;
PATA - Parallel Advanced Technology Attachment;
PCI - Peripheral Component Interconnect;
PCIe - Peripheral Component Interconnect Express;
PCMCIA - Personal Computer Memory Card International Association;
RS-232 - serial interface;
RS-422 - serial interface;
RS-485 - serial interface;
S-100 - S-100 Bus;
SAS - Serial Attached SCSI;
SATA - Serial Advanced Technology Attachment;
SCSI-II - Small Computer System Interface II;
ST506 - Seagate Technology ST-506 Interface;
Thunderbolt - Thunderbolt Interface;
USB - Universal Serial Bus 1.x, 2.x;
VLB - VESA Local Bus

\item
\textbf{\underline{Processors \& Architectures:}} \hspace{0pt}
6502;
6809;
8088;
Alpha (DEC Alpha);
AMD29K;
AMD bit-slice;
ARM (ARMv4–v9);
IBM 360;
Itanium (IA-64);
M68000;
MIPS;
PA-RISC;
PDP-11;
PowerPC (PPC);
RCA COSMAC 1802;
RISC-V;
RS6000;
SPARC;
TMS9900;
VAX;
Z80;
x86 (16-bit 8086);
x86\_32 (IA-32);
x86\_64 (AMD64)

\item
\textbf{\underline{Systems \& Platforms:}} \hspace{0pt}
AIX;
AmigaOS;
Android;
Apollo Domain OS;
BSD (FreeBSD, OpenBSD, NetBSD);
CP/M;
DR-DOS;
FreeRTOS;
HPUX;
IBM 360;
iOS;
Linux (kernel and distros);
macOS;
MINIX;
MPE;
MS-DOS;
Multics;
MTS;
MVS;
MyArc Geneve;
NDOS;
OS/2;
QNX;
RTOS;
SGI IRIX;
SGI ONYX;
Solaris;
TI-99/4A;
Unix;
VOS;
VxWorks;
Windows 3.0;
Windows 3.1;
Windows NT

\item
\textbf{\underline{Drivers \& Low-level Development:}} \hspace{0pt}
Device Tree Overlays (Linux ARM platforms);
HPUX drivers (68020, C);
Linux kernel modules (loadable drivers, LKM);
Linux VFS (Virtual Filesystem Switch);
MPE drivers (Pascal);
MS-DOS INT 13h BIOS-level block device handlers;
macOS drivers;
macOS I/O Kit (C++/Objective-C);
low-level filesystem drivers;
low-level filesystems - XFS and Clustered-XFS;
PCI configuration space access;
RSX-11 and VAX/VMS device drivers (MACRO-11, BLISS);
SCSI-II Autochanger drivers;
SCSI-II Magneto-optic drivers;
Solaris Device Driver Interface (DDI/DKI);
TI-99/4A CRU (Communications Register Unit) drivers (assembly);
Windows 3.x DDK;
Windows DDK (Device Driver Kit);
Windows NT DDK;
BIOS bootloader development;
EFI (Extensible Firmware Interface) programming;
GRUB bootloader configuration and customization;
Secure Boot chain validation and integration.

\item
\textbf{\underline{Programming Languages \& Tools:}}  \hspace{0pt}
.NET;
6502 assembly;
6809 assembly;
8088 assembly;
Alpha assembly;
AMD bit-slice;
AMD29K assembly;
Apollo DSEE (Domain Software Engineering Environment);
ARM assembly;
bash;
bison;
Boost (C++ Libraries);
Borland C++;
BoundsChecker;
C;
C\#;
C++;
C++ template metaprogramming;
Carbon;
ClearCase;
CodeWarrior;
Cucumber;
CVS;
Eclipse (IDE);
Enterprise Architect (Sparx Systems);
flex;
gcc;
gdb;
gem;
Git (Version Control);
Go (Golang);
gprof;
HyperCard;
IDL (Interface Definition Language);
IBM 360 assembly;
IntelliJ IDEA (IDE);
Itanium assembly;
Java;
JavaScript (browser);
JDK (Java Development Kit);
JetBrains MPS;
JNI (Java Native Interface);
JProfiler;
JUnit;
ld;
lex;
Linux kernel modules;
lldb;
M68000 assembly;
macOS I/O Kit;
macOS SDK;
make;
MIPS assembly;
Modern shell (sh, bash, zsh);
MS-DOS INT 13h handlers;
NetBeans (IDE);
Node.js;
npm;
nvm;
NUnit;
OpenAPI (formerly Swagger);
OpenGL;
PA-RISC assembly;
Pascal;
Perforce (Version Control);
Perl;
PDP-11 assembly;
Postman;
PowerPC assembly;
Process Explorer;
Procmon;
Protocol Buffers (protobuf);
PsExec;
Purify;
QTML;
QuickDraw;
QuickTime SDK;
RCS (Version Control);
RCA COSMAC 1802 assembly;
React;
RISC-V assembly;
RSpec;
RS6000 assembly;
Ruby;
Salad;
SCCS (Version Control);
sed;
Selenium;
shell (sh, csh);
SoftICE;
SPARC assembly;
Subversion (SVN Version Control);
Swagger (OpenAPI predecessor);
Sysinternals Suite;
TI GPL (Graphics Programming Language);
TMS9900 assembly;
TypeScript;
UML (Unified Modeling Language);
Unix utilities (awk, sed, grep);
Valgrind;
VAX assembly;
vi;
vim;
Visual Studio (IDE);
VisualVM;
VTune;
webpack;
Win32 API;
Windows 3.x DDK;
Windows DDK;
Windows NT DDK;
Windows SDK;
Windbg;
X11;
Xcode;
XNU kernel;
Xt;
x86 (16-bit 8086) assembly;
x86\_32 (IA-32) assembly;
x86\_64 (AMD64) assembly;
yacc;
yarn;
Z80 assembly
\item
\textbf{\underline{UI \& Graphics:}} \hspace{0pt}
AWT (Abstract Window Toolkit);
Cross-Platform QuickTime;
Direct2D;
Fonts and Typography;
Geometry;
Glyphs;
GTK+;
HTML5 Canvas;
Java2D;
JavaFX;
macOS UI;
Microsoft GDI (Graphics Device Interface);
OpenGL;
Qt;
Quartz 2D;
QuickDraw 2D;
QuickTime 1.x--3.x;
SVG;
Swing;
Windows GUI;
X11;
\item
\textbf{\underline{HID: Graphics, Video\& Sound:}} \hspace{0pt}
DirectX;
DisplayPort;
dropping frames;
FFmpeg;
graphics performance;
H.264;
HDMI;
jitter;
latency;
NVENC;
OpenGL;
sound card drivers;
VGA drivers;
Video Capture;
Video Capture Card Drivers;
Video Capture Cards;
Video Compression;
Video Decompression;
video pipeline;
video transcoding;
WebRTC;
YUV;
\item
\textbf{\underline{Statistical \& Optimization Methods:}} \hspace{0pt}
6-sigma;
A/B testing;
gradient ascent;
gradient boosting;
gradient descent;
K-means clustering;
linear regression;
logistic regression;
monte carlo simulation;
Multidimensional linear regression;
simulated annealing;
stochastic gradient descent;
SVD (Singular Value Decomposition);
\item
\textbf{\underline{Networking \& Protocols:}} \hspace{0pt}
802.11;
BGP;
Bluetooth;
Bluetooth Low Energy;
CAN bus;
DNS;
Ethernet;
FTP;
HTTP;
HTTPS;
IMAP;
IPv4;
IPv6;
ISDN (Integrated Services Digital Network);
leased lines;
LoRaWAN;
LU2;
LU6.2;
Modbus;
MQTT;
Novell NetWare;
POP3;
SMTP;
SNA (System Network Architecture);
SNMP;
SSH;
T1;
TCP/IP;
Telnet;
TLS;
Token Ring;
UDP;
WebSocket;
Wi-Fi;

\item
\textbf{\underline{Enterprise \& Legacy Systems:}} \hspace{0pt}
AS/400;
CICS (Customer Information Control System);
COBOL;
DB2;
HP 3000;
JCL (Job Control Language);
MPE;
MTS (Michigan Terminal System);
MVS (Multiple Virtual Storage);
RSTS;
RSX-11;
System/360;
System/370;
TPF (Transaction Processing Facility);
VAX/VMS;

\item
\textbf{\underline{Project \& Product Experience:}} \hspace{0pt}
Apollo Computer Acquisition;
Apple StarTrek;
Beckman;
BeyondNews;
Bio-Rad;
D. E. Shaw;
EG\&G;
Element Analytics;
F5 Networks;
Fisher Scientific;
Forticom;
GIMPS;
Hewlett Packard;
IdeaLab;
Independa;
Intacct;
Intuit;
NASA;
NASDAQ;
National Merit;
Netflix Prize;
Neustar;
NYSE;
Perkin-Elmer;
Pixar;
Tektronix;
U.S. Coast Guard Auxiliary;
ValueClick;
Varian;
Waters HPLC;

\item
\textbf{\underline{OS Internals \& Concepts:}} \hspace{0pt}
Async distributed software;
bootloaders;
caching
device driver abstraction;
file descriptor management;
Global-scale time synchronization algorithms;
init systems;
interrupt handlers;
kernel scheduling;
memory management;
paging;
preemption;
process isolation;
resource management;
semaphores;
system calls;
virtual memory management;

\item
\textbf{\underline{Development Methodologies:}} \hspace{0pt}
Agile;
Behavior-Driven Development (BDD);
Documentation-driven design;
Kanban;
long locks held across autochanger seeks;
Model-Driven Development (MDD);
product management;
Requirements before Design;
Scrum;
Test-Driven Development (TDD);
Waterfall;

\item
\textbf{\underline{Vintage Computing:}} \hspace{0pt}
5" Floppy (FM/MFM formatting);
8" Floppy (MFM formatting);
Altair 8800;
Apple II;
Apple III;
Commodore Amiga;
Commodore PET;
Commodore SuperPET;
CP/M;
CRT terminals;
DEC PDP-11;
IBM PC XT;
KayPro;
paper tape readers;
S100 bus;
Seequa Chameleon;
Teletypes;
TI-99/4A;
TRS-80;

\item
\textbf{\underline{Language Support \& Internationalization:}} \hspace{0pt}
bidirectional text;
CJK (Chinese, Japanese, Korean) language support;
codepage translation;
collation algorithms;
date and time formatting;
font fallback;
ICU (International Components for Unicode);
ICU collation services;
input method editors (IME);
language negotiation;
locale-aware string sorting;
locale handling;
LTR and RTL layout handling;
multi-script rendering;
natural language collation;
number and currency formatting;
pluralization rules;
sort key generation;
string normalization;
surrogate pairs;
text segmentation;
transliteration;
Unicode code points;
Unicode Internals;
Unicode normalization forms (NFC, NFD, NFKC, NFKD);
Unicode scripts;
UTF-16;
UTF-8;
variant selectors;
zero-width joiners;

\item
\textbf{\underline{Human-Computer Interaction:}} \hspace{0pt}
command-line interfaces;
dialog systems;
eye tracking;
multimodal interfaces;
Speaker-independent speech recognition;
Speech synthesis;
touchscreen UI;
\item
\textbf{\underline{Cloud Scale Platform Engineering:}} \hspace{0pt}
ACME protocol (TLS automation);
Apache Kafka;
Apache Spark;
API backward compatibility;
Attribute-based access control (ABAC);
Blue/green deployments;
Canary deployments;
Centralized log aggregation;
Centralized policy engine (e.g., OPA);
Ceph;
CI/CD enforcement with SAST, DAST, TDD;
Cloud-init;
Cloudflare (CDN);
Code coverage gating;
Consul;
Container image SemVer tagging;
Containerd;
CoreDNS;
CVE scanning policy;
Declarative orchestration (e.g., Kubernetes);
Distributed tracing (OpenTelemetry);
Docker;
ElasticSearch;
etcd;
Feature flag routing;
Federated authentication (OIDC/SAML);
Fine-grained RBAC;
Flannel;
Git commit SHA tagging;
GitOps change management;
gRPC;
Helm;
Horizontal autoscaling policies;
Hyperscalers (AWS, GCP, AZURE);
Immutable audit trails;
Integration test harness;
Istio;
JWT or SPIFFE/SPIRE identity;
Kubernetes (K8s);
Linkerd;
Load and chaos testing;
Log redaction and privacy rules;
Logging access controls;
Logging format: structured JSON;
Metric ontologies (RED, USE);
MinIO;
mTLS enforcement (Istio, Linkerd);
Multi-zone / HA deployment policy;
NGINX;
One process per container;
Open Policy Agent (OPA);
OpenTelemetry;
Policy versioning and rollback;
Production-like staging environments;
Prometheus;
Pulumi;
Redis;
Regression test suite on merge;
Resource quotas and limits;
S3;
SBOM (Software Bill of Materials);
Schema migration strategy;
Secrets management;
Secure Development Lifecycle (SDL);
Secure key vault integration;
Service mesh enforcement;
Session management policies;
Sharding;
SLA-based alerting and load shaping;
SLO dashboards;
SOC-2 Type II compliance;
Structured audit logging;
Synthetic functional tests in production;
Terraform;
TLS cipher suite and protocol policy;
Token lifecycle and revocation;
Trace propagation and sampling;
Vault (HashiCorp);
Versioned architecture standards;
Zero-downtime deployments;
Zookeeper;

\end{itemize}
}

% %-----------INVOLVEMENT---------------
% \section{Leadership / Extracurricular}
%     \resumeSubHeadingListStart
%         \resumeSubheading{Fraternity}{Spring 2020 -- Present}{President}{University Name}
%             \resumeItemListStart
%                 \resumeItem{Achieved a 4 star fraternity ranking by the Office of Fraternity and Sorority Affairs (highest possible ranking).}
%                 \resumeItem{Managed executive board of 5 members and ran weekly meetings to oversee progress in essential parts of the chapter.}
%                 \resumeItem{Led chapter of 30+ members to work towards goals that improve and promote community service, academics, and unity.}
%             \resumeItemListEnd

%     \resumeSubHeadingListEnd


\end{document}
